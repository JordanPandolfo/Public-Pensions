\begin{table}
\centering
\caption{This is caption}
\label{tab:forecast_table}
\begin{tabular}{ccccccccccccc}
\toprule
State &   15y &   30y &   45y &  15y &  30y &  45y &  15y &  30y &  45y & 15y & 30y & 45y \\
\midrule
   AK &    34 &    51 &    60 &   95 &   88 &   74 & 17.1 & 18.5 & 27.2 & 0.9 & 2.1 & 8.1 \\
   AL &    68 &    81 &    89 &   95 &   93 &   87 & 14.2 & 14.4 & 14.8 & 0.4 & 0.6 & 0.9 \\
   AR &    38 &    43 &    47 &   95 &   93 &   89 & 13.3 & 13.4 & 13.7 & 0.4 & 0.5 & 0.7 \\
   AZ &   105 &   135 &   153 &   87 &   84 &   74 & 12.4 & 13.0 & 15.5 & 0.9 & 1.4 & 3.1 \\
   CA &  1439 &  1722 &  1901 &   97 &   94 &   89 & 12.3 & 12.4 & 12.8 & 0.4 & 0.6 & 0.8 \\
   CO &   103 &   136 &   160 &   92 &   90 &   81 & 12.8 & 13.1 & 14.2 & 0.5 & 0.8 & 1.6 \\
   CT &    90 &   106 &   114 &   93 &   93 &   89 & 13.0 & 13.1 & 13.5 & 0.7 & 0.8 & 1.2 \\
   DC &     6 &     5 &     4 &  112 &  116 &  119 &  5.3 &  5.3 &  5.3 & 0.0 & 0.0 & 0.0 \\
   DE &    13 &    16 &    18 &   95 &   92 &   86 & 11.8 & 12.0 & 12.4 & 0.4 & 0.5 & 0.9 \\
   FL &   235 &   269 &   288 &   90 &   87 &   83 &  9.4 &  9.5 &  9.7 & 0.4 & 0.5 & 0.7 \\
   GA &   158 &   201 &   229 &   98 &   95 &   88 & 12.6 & 12.8 & 13.4 & 0.4 & 0.7 & 1.2 \\
   HI &    39 &    49 &    55 &   88 &   87 &   79 & 11.9 & 12.2 & 13.1 & 0.6 & 0.8 & 1.5 \\
   IA &    49 &    56 &    61 &   97 &   96 &   92 & 13.2 & 13.2 & 13.4 & 0.3 & 0.4 & 0.6 \\
   ID &    22 &    28 &    32 &   96 &   93 &   86 & 12.5 & 12.7 & 13.3 & 0.3 & 0.5 & 1.0 \\
   IL &   428 &   515 &   571 &   83 &   86 &   80 & 12.4 & 12.4 & 13.0 & 0.6 & 0.8 & 1.2 \\
   IN &    44 &    52 &    57 &   98 &   96 &   92 & 10.8 & 10.9 & 11.0 & 0.2 & 0.3 & 0.4 \\
   KS &    36 &    43 &    49 &   89 &   87 &   81 & 13.1 & 13.2 & 13.6 & 0.3 & 0.4 & 0.7 \\
   KY &    88 &   105 &   115 &   88 &   91 &   87 & 13.1 & 13.2 & 13.6 & 0.5 & 0.6 & 1.0 \\
   LA &    76 &    88 &    96 &   99 &   97 &   94 & 12.9 & 13.0 & 13.2 & 0.3 & 0.4 & 0.6 \\
   MA &   129 &   149 &   161 &   94 &   93 &   90 & 10.7 & 10.8 & 11.0 & 0.4 & 0.5 & 0.7 \\
   MD &    95 &   112 &   123 &   96 &   94 &   90 & 10.5 & 10.6 & 10.7 & 0.3 & 0.3 & 0.5 \\
   ME &    22 &    27 &    29 &   89 &   85 &   77 & 12.1 & 12.6 & 13.7 & 0.7 & 1.1 & 1.8 \\
   MI &   152 &   180 &   197 &   92 &   91 &   87 & 11.3 & 11.4 & 11.7 & 0.4 & 0.5 & 0.8 \\
   MN &   105 &   131 &   147 &   90 &   85 &   76 & 11.5 & 11.8 & 12.5 & 0.4 & 0.6 & 1.1 \\
   MO &   100 &   118 &   130 &   99 &   97 &   93 & 13.1 & 13.3 & 13.7 & 0.5 & 0.7 & 1.0 \\
   MS &    56 &    68 &    76 &   96 &   93 &   87 & 16.8 & 17.1 & 17.9 & 0.6 & 0.9 & 1.4 \\
   MT &    15 &    18 &    20 &   91 &   91 &   87 & 12.6 & 12.6 & 12.9 & 0.4 & 0.5 & 0.7 \\
   NC &   136 &   164 &   182 &   98 &   95 &   90 & 12.7 & 12.9 & 13.1 & 0.3 & 0.4 & 0.7 \\
   ND &     9 &    10 &    11 &   93 &   94 &   91 & 12.7 & 12.8 & 12.8 & 0.2 & 0.3 & 0.3 \\
   NE &    20 &    23 &    25 &   99 &   98 &   94 & 13.5 & 13.6 & 13.9 & 0.4 & 0.5 & 0.7 \\
   NH &    20 &    25 &    27 &   86 &   83 &   75 & 10.8 & 11.2 & 12.3 & 0.6 & 0.9 & 1.6 \\
   NJ &   201 &   235 &   256 &   61 &   59 &   49 & 11.3 & 11.5 & 12.0 & 0.4 & 0.6 & 0.8 \\
   NM &    56 &    73 &    83 &   83 &   80 &   68 & 18.1 & 19.1 & 23.8 & 1.2 & 1.9 & 4.3 \\
   NV &    79 &   109 &   127 &   89 &   83 &   69 & 10.8 & 11.8 & 17.5 & 0.9 & 1.6 & 4.9 \\
   NY &   651 &   718 &   759 &  101 &  100 &   99 & 12.1 & 12.1 & 12.2 & 0.2 & 0.3 & 0.3 \\
   OH &   301 &   357 &   393 &   96 &   94 &   89 & 12.8 & 13.0 & 13.4 & 0.5 & 0.7 & 1.1 \\
   OK &    45 &    51 &    55 &  102 &  101 &   99 & 14.6 & 14.7 & 14.8 & 0.3 & 0.3 & 0.4 \\
   OR &   102 &   126 &   145 &   95 &   93 &   86 & 12.3 & 12.5 & 13.1 & 0.5 & 0.7 & 1.2 \\
   PA &   198 &   226 &   244 &   82 &   84 &   79 &  9.5 &  9.6 &  9.8 & 0.4 & 0.4 & 0.6 \\
   RI &    18 &    21 &    23 &   93 &   93 &   89 & 10.9 & 11.1 & 11.4 & 0.5 & 0.6 & 0.9 \\
   SC &    70 &    83 &    91 &   94 &   93 &   88 & 13.6 & 13.7 & 14.1 & 0.4 & 0.6 & 0.9 \\
   SD &    15 &    17 &    19 &   99 &   98 &   94 & 12.8 & 12.9 & 13.1 & 0.3 & 0.4 & 0.6 \\
   TN &    65 &    76 &    83 &   98 &   96 &   92 & 10.2 & 10.2 & 10.4 & 0.2 & 0.2 & 0.4 \\
   TX &   438 &   542 &   611 &   98 &   94 &   89 & 11.4 & 11.5 & 11.7 & 0.3 & 0.4 & 0.6 \\
   UT &    44 &    55 &    62 &  100 &   97 &   92 & 12.0 & 12.1 & 12.4 & 0.3 & 0.4 & 0.7 \\
   VA &   135 &   165 &   184 &   92 &   89 &   83 & 11.7 & 11.9 & 12.2 & 0.3 & 0.5 & 0.8 \\
   VT &     9 &    11 &    13 &   89 &   87 &   77 & 12.5 & 12.9 & 14.4 & 0.6 & 0.9 & 1.8 \\
   WA &    99 &   121 &   138 &   97 &   94 &   88 & 12.9 & 13.1 & 13.5 & 0.3 & 0.5 & 0.8 \\
   WI &   145 &   180 &   201 &   95 &   91 &   83 & 12.6 & 13.0 & 13.9 & 0.5 & 0.8 & 1.4 \\
   WV &    21 &    25 &    27 &   95 &   92 &   87 & 16.3 & 16.5 & 17.0 & 0.6 & 0.8 & 1.3 \\
   WY &    12 &    15 &    16 &   89 &   87 &   79 & 17.4 & 17.7 & 18.5 & 0.6 & 0.8 & 1.4 \\
\bottomrule
\end{tabular}
\end{table}
